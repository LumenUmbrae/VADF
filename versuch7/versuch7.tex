\documentclass[Protokollheft.tex]{subfiles}
%\usepackage{amsmath}
\begin{document}
\chapter{HF-Zeitbereich 2: Leitungen und Ports}
%--------------- Start Vorbereitungsaufgaben ---------------

    Mit Hilfe der oben beschriebenen Ansätze sollen die Eigenschaften
    einer einfachen Koaxialleitung untersucht werden. Die Leitung wird
    Ihnen -- bereits diskretisiert -- in zwei Varianten in Form der Materialmatrizen
    $\Meps$ und $\Mmui$ vorgegeben. Im ersten Fall
    handelt es sich um eine homogene Leitung, im zweiten Fall enthält
    die Leitung einen dielektrischen Einsatz (siehe Abbildung~\ref{b1}).
    \begin{figure}[ht]
        \centering
        \begin{subfigure}{0.49\textwidth}
            \centering
            \includegraphics[width=8cm]{v7_homogen.pdf}
        \end{subfigure}
        \begin{subfigure}{0.49\textwidth}
            \centering
            \includegraphics[width=8cm]{v7_inhomogen.pdf}
        \end{subfigure}
        \caption{Verlustfreie Koaxialleitung im Versuch. Es werden zwei Fälle vorgegeben: Eine homogene Koaxialleitung (links) und eine Koaxialleitung mit Einsatz (rechts).}\label{b1}
    \end{figure}
    Die Leitung selbst hat einen quadratischen Querschnitt mit
    ebenfalls quadratischem Innenleiter. Die Kantenlängen des
    Querschnitts betragen $3\,\text{cm}$, beziehungsweise $1\,\text{cm}$, die
    Leitungslänge ist $150\,\text{cm}$.
    Der Einsatz reicht von $75\,\text{cm}$ bis $100\,\text{cm}$.
    Alle Materialien sind als ideal angenommen. Der Innenleiter ist
    perfekt elektrisch leitend, die Dielektrika
    sind verlustfrei, im Fall der homogenen Leitung mit einem Wert von
    $\eps_\text{r}=1.3$, der Einsatz mit $\eps_\text{r}=10$. Die relative Permeabilität beträgt überall $\mur=1$. Der Außenleiter wird durch
    einen, bereits in die vorgegebenen Matrizen eingearbeiteten,
    elektrischen Rand modelliert. Das Grundmaterial wurde so gewählt,
    dass der Wellenwiderstand der Leitung $50\,\Omega$ beträgt.
    
    An den Stirnseiten (vgl. Abbildung~\ref{b3}) wurden, ebenfalls schon in den Matrizen
    enthalten, magnetische Randbedingungen angenommen. Die gesamte
    Struktur wurde mit einem homogenen Gitter mit $\Delta x = \Delta y
    = \Delta z = 1\,\text{cm}$ diskretisiert. Die Struktur enthält damit $4
    \times 4 \times 151$ Gitterpunkte.
    Das Gitter wurde für beide Leitungen identisch gewählt. Alle
    Vorüberlegungen sowie geschriebene Programmroutinen können direkt
    auf die Matrizen beider Strukturen angewendet werden. Als Anregung soll auf der vorderen Stirnseite ein Strom zwischen Innen- und Außenleiter eingeprägt werden. Die entsprechenden Kanten werden also im Laufe des Versuches mit einem Strom angeregt.
    \begin{figure}[h]
        \begin{center}
        \includegraphics[width=8cm]{v7_mesh.pdf}
        \caption{Stirnseite der Koaxialleitung mit Diskretisierungsgitter.}\label{b3}
        \end{center}
    \end{figure}
    \newpage

\section{Vorbereitungsaufgaben}

% --> Aufgabe
\begin{framed}
	\noindent \textbf{1.} Weshalb ist es sinnvoll, für die Stirnseiten der Leitungen
magnetische Randbedingungen zu wählen?\label{exer:bound4frontOfLine}
\end{framed}
Im Falle von Elektrischen Randbedienungen für die magnetische Simulation, bei denen die Normalenkomponente des E-Feldes am Rand Null sind, würde der Rand somit einen PEC darstellt und den Innenleiter mit dem Außenleiter kurzschließen. Zudem würde dies auch zu einem \" verbiegen\" des elektrischen Feldes in der Nähe des Randes führen. \\
Somit sind magnetische Randbedingungen hier die richtige Wahl für die Stirnseiten. 


% --> Aufgabe
\begin{framed}
	\noindent \textbf{2.} Das Gitter ist ein kanonisches kartesisches Gitter. Welchen
Indizes entsprechen diejenigen Kanten in der vorderen und der hinteren
Stirnfläche, die jeweils Innen- und Außenleiter miteinander
verbinden? Welchen Richtungssinn haben sie?\label{exer:idxConductorInterconnection}
\end{framed}

Vom Innen- zum Außenleiter führen in der hinteren Stirnseite die Kanten 7, 11, 2426 und 2427. Vom Außen- zum Innenleiter führen die Kanten 5, 9, 2418 und 2419. \\
In der Vorderen Fläche führen die Kanten 2407, 2411, 4826 und 4827 vom Innen- zur Außenleiter
und die Kanten 2405, 2409, 4818 und 4819 vom Außen- zum Innenleiter. 
% --> Aufgabe
\begin{framed}
	\noindent \textbf{3.} Nehmen Sie die erste Kante in $x$-Richtung, die Innen- und Außenleiter
miteinander verbindet und finden Sie die Indizes der
entsprechenden Kante, jeweils um einen $z$-Gitterschritt nach
hinten versetzt durch alle 151 Ebenen.\label{exer:idxEdge4allZ}
\end{framed}
\noindent
Die erste Kante in $x$-Richtung die Außen- und Innenleiter verbindet ist die Kante 5. \\
Für jeden Versatz um einen Schritt in $z$-Richtung muss 16 Addiert werden. Die Indizes ergeben sich so zu 
\begin{equation}
	E_x = 5+M_z(k-1) = 5 + 16\cdot z.
\end{equation}
Dabei beschreibt $k$ den Index in $z$-Richtung mit $z=k-1$.
% --> Aufgabe
\begin{framed}
	\noindent \textbf{4.} Geben Sie die Funktion eines stückweise linearen
Trapezpulses in Abhängigkeit der Koordinaten der Knickpunkte an.\label{exer:calcPiecewiseTrapezoidal}
\end{framed}

Funktion eines stückweise linearen Trapezpulses, wie in Abb. \ref{tra}  lautet:

$$
f(x)=\left\{
\begin{array}{ll} \frac{t}{t_1}, & x\in [0,t_1) \\
1, & x\in [t_1,t_2) \\
\frac{t_3-t}{t_3-t_2} & x\in [t_2,t_3)\\
0 & x\geq t_3

\end{array}\right. .
$$

\begin{figure}[ht]
	\centering
    \includegraphics[scale=0.9]{v7_trapez.pdf}
    \caption{Trapezpuls}\label{tra}
\end{figure}

% --> Aufgabe
\begin{framed}
	\noindent \textbf{5.} Bestimmen Sie allgemein die Konstanten $\sigma$ und $t_0$
des Gaußpulses in Abhängigkeit der Maximalfrequenz $f_{\text{max}}$ 
und für $t_0$ zusätzlich auch in Abhängigkeit von $\sigma$. Bei
$f=f_{\text{max}}$ soll der Wert des Spektrums genau 1~\% des
Maximalwertes ($\sigma\sqrt{2\pi}$) betragen. Das zugehörige
Zeitsignal soll zum Zeitpunkt $t=0$ nur 0,1~\% seines
Maximalwertes betragen.\label{exer:calcGaussConst}
\end{framed}
\noindent
Die Konstanten $\sigma$ und $t_0$ bestimmt man aus den Formeln des Gauspulses
$$ F(j\omega)=\sigma\sqrt{2\pi}\exp\left(\frac{-\sigma^2\omega^2}{2}\right)\exp\left(-j\omega t_0\right)$$ im Frequenzbereich und
$$
f(t)=\exp\left( \frac{-(t-t_0)^2}{2\sigma^2}         \right)
$$
im Zeitbereich.
\\
\\
Man betrachtet zuerst den Gaußpuls in Zeitbereich. Es ist leicht zu bemerken, dass die Funktion ein Maximum hat. Das Maximum ist 1.
$$f(0)=\exp\left( \frac{-t_0^2}{2\sigma^2}    \right)= 1 \cdot 0.001$$
Daraus erhält man $t_0$.

$$t_0 = \sqrt{6\sigma^2\ln(10)} $$
\\
\\
Jetzt betrachtet man die Formel im Frequenzbereich. Das Maximum dieser Funktion ist $\sigma\sqrt{2\pi}$.

$$ ||F(j\omega_{max})||=\sigma\sqrt{2\pi}\left|\left|\exp\left(\frac{-\sigma^2\omega_{max}^2}{2}\right)\exp\left(-j\omega_{max} t_0\right)\right|\right|=\sigma\sqrt{2\pi}\cdot 0.01$$

Daraus erhält man die Formel
$$\exp\left(\frac{-\sigma^2\omega_{max}^2}{2}\right)=0.01$$
und auch $\sigma$
$$ \sigma =\frac{2\sqrt{\ln(10)}}{\omega_{max}}=\frac{\sqrt{\ln(10)}}{\pi f_{max}}.$$


% --> Aufgabe
\begin{framed}
	\noindent \textbf{6.} Berechnen Sie die maximale Zeitschrittweite nach dem CFL-Kriterium.\label{exer:calcDeltaTmaxCFL}
\end{framed}
\noindent
Mit $\Delta x=\Delta y=\Delta z = 1\,\text{cm}$, sowie der minimalen Permittivität im Rechengebiet $\varepsilon_{min}=1.3\varepsilon_0$ und der minimalen Permeabilität $\mu_{min}=\mu_0$ ergibt sich mit dem CFL-Kriterium
\begin{eqnarray*}
	\Delta t\leq\Delta t^{\text{CFL}}_{max}&=&\min_j\Bigg\{\sqrt{\varepsilon_j\mu_j}\sqrt{\frac{1}{\frac{1}{\Delta x^2}+\frac{1}{\Delta y^2}+\frac{1}{\Delta z^2}}}\Bigg\}\\
	\Delta t^{\text{CFL}}_{max}&=&\sqrt{1.3\varepsilon_0\mu_0}\sqrt{\frac{1\,\text{cm}^2}{3}}\\
	\Delta t^{\text{CFL}}_{max}&=&2.196\cdot 10^{-11}\,\text{s}
\end{eqnarray*}

\section{Aufgaben während der Praktikumssitzung}
Die vorliegende Beschreibung soll nur die grobe Vorgehensweise
während des Versuchs vorgeben. Sind die grundlegenden Dinge wie
Pulsgenerierung, modifizierter Leapfrog und visuelle Darstellung
implementiert, lassen sich verschiedene Pulsformen, verschiedene
Anregungen, Leitungsabschlüsse und die beiden Leitungen beliebig
miteinander kombinieren. Prinzipiell soll die Fortpflanzung eines
elektrischen Pulses über die vorgegebene Struktur visualisiert
werden. Dazu werden in jedem Zeitschritt die elektrischen Spannungen an hintereinander
liegenden Kanten über die z-Achse aufgetragen. Zum
selbstständigen Experimentieren soll hierbei durchaus ermutigt
werden. Der Hauptbestandteil der Implementierung in diesem Versuch soll in dem Skript \verb\versuch7.m\ erfolgen.

% --> Aufgabe
\begin{framed}
	\noindent \textbf{1.} Verwenden Sie die Leapfrog-Routine aus dem letzten Versuch. Nutzen Sie hierfür die bereits teilweise gegebene Methode \verb\leapfrog.m\, indem Sie den Eingabe-Parameter \verb\Rmat\ zunächst ignorieren.
Regen Sie eine beliebige Kante, die auf der vorderen Stirnfläche Innen-
und Außenleiter verbindet, an. Anregungssignal soll ein Trapezpuls
mit Anstiegs- und Abfallzeit $t_1=t_3-t_2=\SI{0,5}{ns}$ und einer
Haltezeit $t_2-t_1=\SI{0,7}{ns}$ sein. Es sollen zunächst 1000 Zeitschritte
berechnet werden. Speichern sie die elektrische Spannung der
von Ihnen ausgewählten Kante in jeder der 151 Ebenen ab. Plotten Sie damit
das elektrische Feld zwischen Innen- und Außenleiter in
Abhängigkeit von $z$ und verfolgen Sie den Verlauf über die Zeit
(als Film) mithilfe des {\texttt{drawnow}}-Befehls.\label{exer:visualizeTLine}
%
\end{framed}
\noindent
Für die Anregung einer einzelnen Kante wurde die Kante mit Index 5 gewählt. Die Ausbreitung vor der Reflexion ist in Abbildung \ref{fig:Aufg1_1} zu sehen, die Ausbreitung nach der Reflexion in Abbildung \ref{fig:Aufg1_2}. Deutlich zu erkennen ist ein Rauschen, dass mit Trapezpuls überlagert ist. Der Puls wird vorzeichenerhaltend reflektiert.
\begin{figure}[ht]
	\centering
	\includegraphics[trim = 20mm 65mm 20mm 65mm, clip,width=0.7\linewidth]{Aufgabe1.pdf}
	\caption{Trapezpuls vor der Reflexion ohne Leitungsabschluss.}\label{fig:Aufg1_1}
\end{figure}
\begin{figure}[ht]
	\centering
	\includegraphics[trim = 20mm 65mm 20mm 65mm, clip,width=0.7\linewidth]{Aufgabe1_2.pdf}
	\caption{Trapezpuls nach der Reflexion ohne Leitungsabschluss.}\label{fig:Aufg1_2}
\end{figure}

% --> Aufgabe
\begin{framed}
	\noindent \textbf{2.} Variieren Sie Ihre Leapfrog-Routine so, dass sie im
Folgenden auch konzentrierte Elemente berücksichtigen kann. Aus der Leitungstheorie 
ist bekannt, dass der Reflexionsfaktor $\Gamma$ bei einem Abschluss $Z_{\text{2}}$ am Ende der Leitung gerade 
\begin{align}
 \Gamma=\frac{Z_{\text{2}}-Z_{\text{w}}}{Z_{\text{2}}+Z_{\text{w}}},
\end{align}
mit dem Wellenwiderstand $Z_{\text{w}}$ beträgt. Für einen reflexionsfreien Abschluss soll die 
Leitung hier also mit ihrem Wellenwiderstand abgeschlossen werden,
indem einer der Kanten in der hinteren Stirnfläche ein Widerstand
von \SI{50}{\ohm} zugeordnet wird. Die Anzahl der Zeitschritte kann nach
eigenem Ermessen verringert werden. Der Durchlauf des Pulses durch
die Leitung soll auch in den weiteren Aufgabenteilen als Film
betrachtet werden.\label{exer:simIncludeLumped}\\
\textbf{Hinweis:} Benutzen Sie zum Berechnen der inversen Matrix 
in~(7.6) den Befehl \verb|nullInv|.
%
\end{framed}
\noindent
Für den Abschluss mit einem Widerstand von $50\,\Omega$ wurde die Kante mit dem Index 2405 gewählt. Die Ausbreitung vor der Reflexion ist in Abbildung \ref{fig:Aufg2_1} zu sehen, die Ausbreitung nach der Reflexion in Abbildung \ref{fig:Aufg2_2}. Es ist zu erkennen, dass der Puls komplett absorbiert wird, jedoch noch Restreflexionen auftreten.


\begin{figure}[ht]
	\centering
	\includegraphics[trim = 20mm 65mm 20mm 65mm, clip,width=0.7\linewidth]{Aufgabe2_1.pdf}
	\caption{Trapezpuls vor der Reflexion mit einem Leitungsabschluss an einer Kante von $50\,\Omega$.}\label{fig:Aufg2_1}
\end{figure}
\begin{figure}[ht]
	\centering
	\includegraphics[trim = 20mm 65mm 20mm 65mm, clip,width=0.7\linewidth]{Aufgabe2_2.pdf}
	\caption{Trapezpuls nach der Reflexion mit einem Leitungsabschluss an einer Kante von $50\,\Omega$.}\label{fig:Aufg2_2}
\end{figure}

% --> Aufgabe
\begin{framed}
	\noindent \textbf{3.} Der Leitungsabschluss kann verbessert werden, indem der Gesamtwiderstand 
auf die acht Kanten verteilt wird. Vergleichen Sie das Reflexionsverhalten
mit dem in der vorherigen Teilaufgabe. Erklären Sie die Verbesserung!\label{exer:distributeTermination}
\end{framed}
\noindent
Mit Gegensatz zur vorherigen Aufgabe konnte durch die Verteilung auf die 8 Endkanten die Restreflexionen deutlich verringert werden. Der Puls vor der Reflexion ist in Abbildung \ref{fig:Aufg3_1} zu sehen, die Verbesserung nach der Reflexion in Abbildung \ref{fig:Aufg3_2}.
\begin{figure}[ht]
	\centering
	\includegraphics[trim = 20mm 65mm 20mm 65mm, clip,width=0.7\linewidth]{Aufgabe3_1.pdf}
	\caption{Trapezpuls vor der Reflexion mit einem Leitungsabschluss von $50\,\Omega$ mit 8 Kanten.}\label{fig:Aufg3_1}
\end{figure}
\begin{figure}[ht]
	\centering
	\includegraphics[trim = 20mm 65mm 20mm 65mm, clip,width=0.7\linewidth]{Aufgabe3_2.pdf}
	\caption{Trapezpuls nach der Reflexion mit einem Leitungsabschluss von $50\,\Omega$ mit 8 Kanten.}\label{fig:Aufg3_2}
\end{figure}

% --> Aufgabe
\begin{framed}
	\noindent \textbf{4.} Nun soll auch die Anregung symmetrisiert werden. Teilen Sie
den Anregungsstrom auf die acht Kanten der vorderen Stirnfläche
auf. Schließen Sie auch den vorderen Port reflexionsfrei ab.
Zeigen Sie die Verbesserung durch diese Maßnahme.\label{exer:symmetrizeExcitation}
\end{framed}
\noindent
Durch die Verteilung des Anregestroms, sowie durch den reflexionsfreien Abschluss des vorderen Ports, können Rauschen und Oberwellen fast komplett eliminiert werden, was aus Abbildung \ref{fig:Aufg4_1} deutlich wird. Deshalb tritt nun nach der Absorption auch kaum noch Rauschen auf, was in Abbildung \ref{fig:Aufg4_2} zu sehen ist.

\begin{figure}[ht]
	\centering
	\includegraphics[trim = 20mm 65mm 20mm 65mm, clip,width=0.7\linewidth]{Aufgabe4_1.pdf}
	\caption{Trapezpuls vor der Reflexion mit einem Leitungsabschluss von $50\,\Omega$ mit 8 Kanten, sowie Verteilung des Anregestroms und reflexionsfreiem Abschluss des vorderen Ports.}\label{fig:Aufg4_1}
\end{figure}
\begin{figure}[ht]
	\centering
	\includegraphics[trim = 20mm 65mm 20mm 65mm, clip,width=0.7\linewidth]{Aufgabe4_2.pdf}
	\caption{Trapezpuls nach der Reflexion mit einem Leitungsabschluss von $50\,\Omega$ mit 8 Kanten, sowie Verteilung des Anregestroms und reflexionsfreiem Abschluss des vorderen Ports.}\label{fig:Aufg4_2}
\end{figure}




% --> Aufgabe
\begin{framed}
	\noindent \textbf{5.} Verwenden Sie nun anstelle des Trapezpulses einen Gaußpuls
mit $f_{\text{max}}=\SI{1}{GHz}$. Was fällt bei der Ausbreitung im Vergleich
zum Trapezpuls auf?\label{exer:gaussExcitation}
\end{framed}
\noindent
Die Vorgang läuft mit dem Gaußpuls ganz ähnlich ab wie mit dem Trapezpuls, nur, dass von Anfangan kein Rauschen vorhanden ist. In Abbildungen  \ref{fig:gauss1},\ref{fig:gauss2},\ref{fig:gauss3}, \ref{fig:gauss4}, \ref{fig:gauss5} und \ref{fig:gauss6} sind diese Fälle dargestellt.
\begin{figure}[ht]
	\centering
	\includegraphics[trim = 20mm 65mm 20mm 65mm, clip,width=0.7\linewidth]{untitled1.pdf}
	\caption{Gaußpuls vor der Reflexion ohne Leitungsabschluss.}\label{fig:gauss1}
\end{figure}
\begin{figure}[ht]
	\centering
	\includegraphics[trim = 20mm 65mm 20mm 65mm, clip,width=0.7\linewidth]{untitled2.pdf}
	\caption{Gaußpuls nach der Reflexion ohne Leitungsabschluss.}\label{fig:gauss2}
\end{figure}
\begin{figure}[ht]
	\centering
	\includegraphics[trim = 20mm 65mm 20mm 65mm, clip,width=0.7\linewidth]{untitled3.pdf}
	\caption{Gaußpuls vor der Reflexion mit einem Leitungsabschluss an einer Kante von $50\,\Omega$.}\label{fig:gauss3}
\end{figure}
\begin{figure}[ht]
	\centering
	\includegraphics[trim = 20mm 65mm 20mm 65mm, clip,width=0.7\linewidth]{untitled4.pdf}
	\caption{Gaußpuls nach der Reflexion mit einem Leitungsabschluss an einer Kante von $50\,\Omega$.}\label{fig:gauss4}
\end{figure}
\begin{figure}[ht]
	\centering
	\includegraphics[trim = 20mm 65mm 20mm 65mm, clip,width=0.7\linewidth]{untitled5.pdf}
	\caption{Gaußpuls vor der Reflexion mit einem Leitungsabschluss von $50\,\Omega$ mit 8 Kanten.}\label{fig:gauss5}
\end{figure}
\begin{figure}[ht]
	\centering
	\includegraphics[trim = 20mm 65mm 20mm 65mm, clip,width=0.7\linewidth]{untitled6.pdf}
	\caption{Gaußpuls nach der Reflexion mit einem Leitungsabschluss von $50\,\Omega$ mit 8 Kanten.}\label{fig:gauss6}
\end{figure}


% --> Aufgabe
\begin{framed}
	\noindent \textbf{6.} Verwenden Sie außer der homogenen Leitung nun auch die
inhomogene. Beachten Sie die Pulsform und
Ausbreitungsgeschwindigkeit innerhalb des dielektrischen
Einsatzes.\label{exer:inhomogenTLine}
\end{framed}
\noindent
Nach den Simulationen eines homogenen Leiter wird nun auch der inhomogene Leiter simuliert. Als inhomogenen Leiter betrachtet man den Leiter, der von 75cm bis 100cm in Ausbreitungsrichtung einen dielektrischen Einsatz von $\eps_r=10$ hat. In den Abbildungen \ref{fig:inh1},\ref{fig:inh2},\ref{fig:inh3},\ref{fig:inh4},\ref{fig:inh5} und \ref{fig:inh6} sind die Simulationen dargestellt. Es wird aus den Abilldungen deutlich, dass an den Übergängen zum jeweils anderen Material Reflexionen und Transmissionen entstehen. Dabei werden die Amplituden der Pulse immer kleiner.\\ \\
Jedes Puls(Trapezpuls oder Gausspuls) wird innerhalb des dielektrisches Einsatzes verkleinert und verlangsamt.

\begin{figure}[ht]
	\centering
	\includegraphics[trim = 15mm 65mm 20mm 65mm, clip,width=0.7\linewidth]{untitledT1.pdf}
	\caption{Trapezpuls vor der Reflektion am Einsatz ohne Leitungsabschluss}\label{fig:inh1}
\end{figure} 
\begin{figure}[ht]
	\centering
	\includegraphics[trim = 15mm 65mm 20mm 65mm, clip,width=0.7\linewidth]{untitledT2.pdf}
	\caption{Trapezpuls im Einsatz bei einem Leitungsabschluss von $50\,\Omega$.}\label{fig:inh2}
\end{figure} 
\begin{figure}[ht]
	\centering
	\includegraphics[trim = 15mm 65mm 20mm 65mm, clip,width=0.7\linewidth]{untitledT3.pdf}
	\caption{Trapezpuls nach weiteren Reflexionen}\label{fig:inh3}
\end{figure} 



\begin{figure}[ht]
	\centering
	\includegraphics[trim = 15mm 65mm 20mm 65mm, clip,width=0.7\linewidth]{untitledG1.pdf}
	\caption{Gaußpuls vor der Reflektion am Einsatz ohne Leitungsabschluss}\label{fig:inh4}
\end{figure} 
\begin{figure}[ht]
	\centering
	\includegraphics[trim = 15mm 65mm 20mm 65mm, clip,width=0.7\linewidth]{untitledG2.pdf}
	\caption{Gaußpuls im Einsatz bei einem Leitungsabschluss von $50\,\Omega$}\label{fig:inh5}
\end{figure} 
\begin{figure}[ht]
	\centering
	\includegraphics[trim = 15mm 65mm 20mm 65mm, clip,width=0.7\linewidth]{untitledG3.pdf}
	\caption{Gaußpuls nach weiteren Reflexionen}\label{fig:inh6}
\end{figure} 






\section{Fazit}
Durch diesen Versuch ist es nun möglich Ports und konzentrierte Elemente zu modellieren. Dadurch kann z. B. das Reflexionsverhalten besser simuliert und analysiert werden.

\end{document}
