\documentclass[Protokollheft.tex]{subfiles}
\begin{document}
\chapter{HF-Zeitbereich 3: Streuparameter}
%--------------- Start Vorbereitungsaufgaben ---------------
Dieser Versuch schließt direkt an den letzten Versuch an. Es soll dieselbe
Problemstellung sowie das im letzten Versuch bestimmte Zeitsignal verwendet werden,
um dann mithilfe der DFT Analysen im Frequenzbereich durchzuführen. Insbesondere 
werden die Anregung $I_0$ sowie die Abschlusswiderstände wieder über die jeweilige 
Stirnfläche verteilt. Beachten Sie aber, dass im Gegensatz zum letzten Versuch hier nun
für die homogene Leitung eine relative Permittivität von $\epsr=0.9$ verwendet wird.
Dieser Wert ist zwar unphysikalisch, stellt aber sicher, dass der 
Leitungswellenwiderstand \SI{50}{\ohm} beträgt. Für die inhomogene Leitung gilt weiterhin
$\epsr=1.3$, während für beide Leitungen weiterhin $\mur=1$ gewählt wurde.

\section{Vorbereitungsaufgaben}

% --> Aufgabe
\begin{framed}
	\noindent \textbf{1.} Zeichnen Sie für die hier behandelte Leitung das entsprechende Zweitor mit Zählpfeilen orientiert wie in Abb.~7.7. Fügen Sie auch die hier verwendete äußere Beschaltung an den Leitungsenden hinzu.\label{exer:twoPort}
\end{framed}

\emph{Fügen Sie hier Ihre Lösung ein}

% --> Aufgabe
\begin{framed}
	\noindent \textbf{2.} Bestimmen Sie den Eingangsstrom $I_1$ in Abhängigkeit von der Eingangsspannung $U_1$, dem Abschlusswiderstand $R$ und der Anregung $I_0$.\label{exer:calcI1}
\end{framed}

\emph{Fügen Sie hier Ihre Lösung ein}

% --> Aufgabe
\begin{framed}
	\noindent \textbf{3.} Bestimmen Sie die Ausgangsspannung $\ul{U}_2$ und den Ausgangsstrom $\ul{I}_2$ in Abhängigkeit von der Eingangsspannung $U_1$, der Länge der Leitung $\ell$ und der Phasenkonstante $\beta$. Sie können dabei annehmen, dass die Leitung mit ihrem Wellenwiderstand abgeschlossen ist.\label{exer:calcU2I2}
\end{framed}

\emph{Fügen Sie hier Ihre Lösung ein}

% --> Aufgabe
\begin{framed}
	\noindent \textbf{4.} Mit welcher Geschwindigkeit bewegt sich die Welle auf der gegebenen Leitung? Berechnen Sie die Zeit, die die Welle benötigt, um die Länge der Leitung einmal zu passieren.\label{exer:calcSpeedTime}
\end{framed}

\emph{Fügen Sie hier Ihre Lösung ein}

% --> Aufgabe
\begin{framed}
	\noindent \textbf{5.} Wie kann man aus den im Allgemeinen komplexen Rückgabewerten der DFT (\matlab-Befehl \verb"fft") auf das Frequenzspektrum schließen?\label{exer:freqSpectByDFT}
\end{framed}

\emph{Fügen Sie hier Ihre Lösung ein}

\section{Aufgaben während der Praktikumssitzung}

{\subsection{Streuparameter}}

\noindent
Im Folgenden soll das Übertragungsverhalten der beiden Leitungen
im Frequenzbereich von $0$ bis \SI{200}{MHz} bestimmt werden. Zu diesem
Zweck sollen nur noch Gaußpulse verwendet werden.

% --> Aufgabe
\begin{framed}
	\noindent \textbf{1.} Regen Sie die homogene Leitung mit dem in Versuch 7 beschriebenen
Gauß-Puls an.\label{exer:exciteGauss}
\end{framed}

\emph{Fügen Sie hier Ihre Lösung ein}

% --> Aufgabe
\begin{framed}
	\noindent \textbf{2.} Bestimmen Sie die Spannungen und Ströme an Ein- und Ausgang der Leitung
 im Zeitbereich und stellen Sie diese in entsprechenden Plots dar.\label{exer:UandVtimeDomain}
\end{framed}

\emph{Fügen Sie hier Ihre Lösung ein}

% --> Aufgabe
\begin{framed}
	\noindent \textbf{3.} Schreiben Sie eine Routine, die das Spektrum eines
Zeitsignals berechnet und eine zugehörige Frequenzachse erzeugt.
Experimentieren Sie mit dem zero-padding, um im interessierenden
Frequenzbereich eine genügend gute Auflösung zu bekommen.\label{exer:calcFreqSpectWithAxis}
\end{framed}

\emph{Fügen Sie hier Ihre Lösung ein}

% --> Aufgabe
\begin{framed}
	\noindent \textbf{4.} Bestimmen Sie die Spannungen und Ströme an Ein- und Ausgang der Leitung 
im Frequenzbereich und stellen Sie diese in entsprechenden Plots dar. Überprüfen Sie, ob 
der Gauß-Puls im Frequenzbereich die Bedingungen erfüllt, die an 
diesen in Versuch 7 gestellt wurden.\label{exer:UandVfreqDomain}
\end{framed}

\emph{Fügen Sie hier Ihre Lösung ein}

% --> Aufgabe
\begin{framed}
	\noindent \textbf{5.} Berechnen Sie die Ein- und Ausgangsimpedanz im Frequenzbereich und stellen 
Sie diese in Abhängigkeit der Frequenz dar.\label{exer:ZfreqDomain}
\end{framed}

\emph{Fügen Sie hier Ihre Lösung ein}

% --> Aufgabe
\begin{framed}
	\noindent \textbf{6.} Berechnen Sie aus den Spektren der Strom- und 
Spannungsgrößen die Spektren der zugehörigen Wellengrößen $a_1$,
$b_1$ und $b_2$ und daraus die Streuparameter $S_{11}$ und
$S_{21}$. Interpretieren Sie das Ergebnis für Reflexion und
Transmission.\label{exer:calcWaveQuantities}
\end{framed}

\emph{Fügen Sie hier Ihre Lösung ein}

% --> Aufgabe
\begin{framed}
	\noindent \textbf{7.} Überprüfen Sie die Energiebilanz nach~(8.8).\label{exer:checkEnergyBal4TLine}
\end{framed}

\emph{Fügen Sie hier Ihre Lösung ein}

% --> Aufgabe
\begin{framed}
	\noindent \textbf{8.} Wiederholen Sie die Berechnungen für die inhomogene Leitung.\label{exer:calc4inhomTLine}
\end{framed}

\emph{Fügen Sie hier Ihre Lösung ein}

{\subsection{Lösung im Frequenzbereich}}

% --> Aufgabe
\begin{framed}
	\noindent \textbf{9.} Berechnen Sie das System für einen Frequenzpunkt im
Frequenzbereich, wie in Versuch 5 beschrieben. Die zugehörige
Gleichung für den verlustlosen Fall mit konzentrierten
Elementen lautet
\begin{equation}
	\underbrace{(\curldfit \Mmui \curlfit -
	\omega^2 \Meps + j \omega {\textbf{R}}^{-1} )}_{\Amat} \ul \efit = - j \omega \ul \jfit_{\text{e}}.
\end{equation}
Vergleichen Sie die Rechenzeit dieses einen Punktes mit einem
kompletten Leapfrog-Durchlauf.\label{exer:cmpFreqSolWithLeapfrog}
\end{framed}

\emph{Fügen Sie hier Ihre Lösung ein}



\section{Fazit}
\emph{Fügen Sie hier Ihre Lösung ein}

\end{document}
