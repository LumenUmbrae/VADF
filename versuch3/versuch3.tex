\documentclass[Protokollheft.tex]{subfiles}
\begin{document}
\chapter{Grundlagen der Methode der Finiten Integration 2}
%--------------- Start Vorbereitungsaufgaben ---------------

\section{Vorbereitungsaufgaben}

% --> Aufgabe
\begin{framed}
	\noindent \textbf{1.} Überlegen Sie sich, wie man ausgehend vom 3-fach Index $i,j,k$ (vgl. Gl.~(3.1)) die Randpunkte eines kartesischen Rechengebietes im kanonischen Indizierungsschema bestimmt (eine Skizze ist hilfreich). Schreiben Sie hierfür ein Schleifenkonstrukt in Pseudocode.\label{exer:boundIdx}
\end{framed}

\emph{Fügen Sie hier Ihre Lösung ein}

% --> Aufgabe
\begin{framed}
	\noindent \textbf{2.} Wie sehen für ein äquidistantes, kartesisches Gitter die Geometriematrizen $\DS$, $\DSd$, $\DA$ und $\DAd$ aus? Was ist bei den Rändern zu beachten? Welche Dimensionen besitzen die Matrizen?\label{exer:geoMatsStructure}
\end{framed}

\emph{Fügen Sie hier Ihre Lösung ein}

% --> Aufgabe
\begin{framed}
	\noindent \textbf{3.} Skizzieren Sie kurz, wie sich die Materialmatrizen zusammenstellen. Wie sind hierbei die Randbedingungen (elektrisch \& magnetisch) einzuarbeiten bzw. muss überhaupt eine Änderung vorgenommen werden?\label{exer:materialMats}
\end{framed}

\emph{Fügen Sie hier Ihre Lösung ein}

% --> Aufgabe
\begin{framed}
	\noindent \textbf{4.} Um die im Versuch zu implementierende Visualisierung zu testen, soll ein vorgegebenes
    rotationssymmetrisches Feld in Zylinderkoordinaten nach der analytischen Formel
    \begin{align}
     \vec{D}(r,\varphi,z)=\frac{1}{r^2} \er
    \end{align}
    visualisiert werden. Es soll ein äquidistantes Gitter benutzt werden, dessen Mitte genau dem Koordinatenursprung entspricht.\\
Bestimmen Sie die diskreten Größen $\dfitloc(n)$ und $\efitloc(n)$ des vorgegebenen Feldes. Zur Vereinfachung soll bei der hierfür notwendigen Integration der Feldwert in der Mitte der Strecke bzw. Fläche als repräsentativ gelten und damit als konstant über dem gesamten Element angenommen werden. \\
    \ \\
    {\textbf{Hinweis:}} Transformieren Sie zuerst zur Bestimmung der notwendigen Feldwerte das gegebene Feld in kartesische Koordinaten $\vec{D}(x,y,z)$.\label{exer:visualizeFieldPrep}
%
\end{framed}

\emph{Fügen Sie hier Ihre Lösung ein}



\section{Aufgaben während der Praktikumssitzung}

\subsection{Materialmatrizen}

% --> Aufgabe
\begin{framed}
	\noindent \textbf{1.} Zuerst sollen zwei Funktionen zum Bestimmen der Geometriematrizen $\DS$, $\DSd$ und $\DA$ geschrieben werden:
\begin{align}
\lstinline{[DS, DSt]} &= \lstinline{createDS(msh)} \label{pro:createDs}\\
\lstinline{[DA]} &= \lstinline{createDA(DS)} \label{pro:createDA}
\end{align}
Wie kann mit der zweiten Funktion auch $\DAd$ bestimmt werden?\label{exer:createDS_DA}
\end{framed}

\emph{Fügen Sie hier Ihre Lösung ein}

% --> Aufgabe
\begin{framed}
	\noindent \textbf{2.} Nun sollen die Funktionen
\begin{align}
\lstinline{[Deps]} &= \lstinline{createDeps(msh, DA, DAt, eps_r, bc)} \label{pro:createDeps}\\
\lstinline{[Meps]} &= \lstinline{createMeps(DAt, Deps, DS)} \label{pro:createMeps}
\end{align}
geschrieben werden, um die $\Meps$-Matrix \lstinline{Meps} aus der $\Deps$-Matrix \lstinline{Deps}
der gemittelten Permittivitäten zu bestimmen.
\lstinline{bc} $=1$ soll dabei elektrische und \lstinline{bc} $=2$ magnetische Randbedingungen bedeuten.
Die Materialverteilung auf dem Gitter \lstinline{msh} soll inhomogen und isotrop bezüglich der Raumrichtungen sein. Zur besseren Übersicht sollen bei der Übergabe relative Permittivitäten verwendet werden. \lstinline{eps_r} soll damit als $N_\text{P}\times 1$ Matrix übergeben werden, also für jedes der $N_\text{P}$ primären Volumen ein $\eps_\text{r}$-Wert.
\\
\textbf{Hinweis:} Für das Invertieren von $\DS$ ist die Methode
\lstinline{nullInv} vorgegeben.\label{exer:createDeps_Meps}
\end{framed}

\emph{Fügen Sie hier Ihre Lösung ein}

% --> Aufgabe
\begin{framed}
	\noindent \textbf{3.} Die Funktion~\eqref{pro:createMeps} soll nun mit den Parametern
\lstinline{xmesh = [-2 0 2]}, \lstinline{ymesh = [-1 0 1]}, \lstinline{zmesh = [0 1]} und
isotropem $\eps=\eps_0$ die Materialmatrix
$\Meps$ für elektrische Randbedingungen berechnen und ausgeben. Vervollständigen Sie hierfür das bereits gegebene Skript \lstinline{exampleMeps.m}\label{exer:MepsExample}
\end{framed}

\emph{Fügen Sie hier Ihre Lösung ein}

\subsection{Interpolation und Visualisierung}

% --> Aufgabe
\begin{framed}
	\noindent \textbf{4.} Programmieren Sie eine Routine
\begin{align}
	\lstinline{eField = fitInt(msh, eBow)} \; , \label{mthd:fitInt}
\end{align}	
die die Komponenten von $\efit$ als $\vec{E}$-Feld auf die primären Punkte interpoliert.\label{exer:fitInt}
\end{framed}

\emph{Fügen Sie hier Ihre Lösung ein}

% --> Aufgabe
\begin{framed}
	\noindent \textbf{5.} Schreiben sie eine Methode
\begin{align}
	\lstinline{plotEBow(msh, eBow, indz)} \; , \label{mthd:plotEBow}
\end{align}
die auf Methode~\eqref{mthd:fitInt} aufbauend $\efit$ interpoliert und den
Betrag des $\vec{E}$-Feldes mit dem \matlab-Befehl \lstinline{surf} in einer
$x$-$y$-Ebene mit Index \lstinline{indz} grafisch darstellt. Verwenden Sie hierfür bitte elektrische Randbedingungen.
\\
\textbf{Hinweis:} Nutzen Sie auch für das Invertieren von $\Meps$ die vorgegebene Methode
\lstinline{nullInv}.\label{exer:plotEBow}
\end{framed}

\emph{Fügen Sie hier Ihre Lösung ein}

% --> Aufgabe
\begin{framed}
	\noindent \textbf{6.} Geben Sie das rotationssymmetrische Feld aus der Vorbereitung als
Vektor $\dfit$ vor, berechnen Sie daraus mit Hilfe der Materialmatrix $\Mepsi$
das Feld $\efit$ und wenden Sie dann Methode \eqref{mthd:plotEBow} an. Visualisieren Sie außerdem die selbe Schnittebene mit der in Versuch 2 vorgestellten Methode \lstinline{plotEdgeVoltage}. Vervollständigen Sie hierfür den ersten Teil des bereits gegebenen Skripts \lstinline{exampleVisualEfield.m}\label{exer:exampleVisualEfield1}
\end{framed}

\emph{Fügen Sie hier Ihre Lösung ein}

% --> Aufgabe
\begin{framed}
	\noindent \textbf{7.} Überlegen Sie sich, welche Änderungen an den bisher implementierten Methoden 
vorgenommen werden müssen, um ein anisotropes Material zu verwenden. Ändern Sie 
Ihre Implementierung entsprechend und verwenden Sie ein anisotropes Material mit unterschiedlichen
Permittivitäten in $x$- und $y$-Richtung (z.\,B.
$\eps_x / \eps_y=4$) sowie elektrische
Randbedingungen. Interpolieren und visualisieren Sie das Feld
$\efit$ wie in der Aufgabe zuvor. Visualisieren Sie auch hier das Ergebnis zusätzlich mit der Methode \lstinline{plotEdgeVoltage}. Vervollständigen Sie hierfür den zweiten Teil des bereits gegebenen Skripts \lstinline{exampleVisualEfield.m}\label{exer:exampleVisualEfield2}\\
\end{framed}

\emph{Fügen Sie hier Ihre Lösung ein}



\section{Fragen zur Ausarbeitung}

% --> Aufgabe
	\begin{framed}
	\noindent \textbf{1.} Erstellen Sie eine 2D-Skizze einer dualen
	Gitterfläche mit den zugehörigen primären Gitterzellen, welche
    zur Mittelung der Permittivität notwendig sind (siehe (3.10)).\label{exer:averagingEps}
\end{framed}

\emph{Fügen Sie hier Ihre Lösung ein}

% --> Aufgabe
	\begin{framed}
	\noindent \textbf{2.} Häufig werden für die Visualisierung der magnetischen Feldstärke
	$\Hv$ die entsprechenden Komponenten ebenfalls auf den Punkten des
	primären Gitters gemittelt und nicht auf den dualen
	Punkten. Beschreiben Sie für diese Mittelung \emph{kurz} eine geeignete Vorgehensweise (kleine Skizze sinnvoll)
	und gehen Sie dabei auch auf die Randbedingungen ein.\label{exer:averageHfield}
\end{framed}

\emph{Fügen Sie hier Ihre Lösung ein}





\section{Fazit}
\emph{Fügen Sie hier Ihre Lösung ein}

\end{document}
